% Copyright 2015 by Kevin Schmidiger
% mix of classic and bank style of https://www.sharelatex.com/templates/cv-or-resume
% !TEX program = pdflatex

\documentclass[11pt,legalpaper,sans]{moderncv}   
% ('10pt', '11pt' and '12pt'), ('a4paper', 'letterpaper', 'a5paper', 'legalpaper', 'executivepaper' or 'landscape') , ('sans' or 'roman')

% ModernCV Themes
\usepackage{moderncvstylemyclassic}
%\moderncvstyle{myclassic}                        % 'casual' (default),'classic', 'oldstyle' or 'banking'
\moderncvcolor{blue}                          % 'blue', 'orange', 'green', 'red', 'purple', 'grey' or 'black'
%\nopagenumbers{}

% ajustes para los margenes de pagina
\usepackage[scale=0.75]{geometry}
%\setlength{\hintscolumnwidth}{3cm}

\usepackage[utf8]{inputenc}					% Added so I can use special letters like an "Ö"
\usepackage{german}

\setlength{\headheight}{0pt}
\setlength{\textheight}{850pt}

% Personal data
\name{Kevin}{Schmidiger}
\title{Lebenslauf}
\address{Arbentalstrasse 317, 8045 Z"urich}
\phone[mobile]{+41 79 269 82 25}
%\phone[fixed]{+41 44 341 98 27}
%\phone[fax]{+3~(456)~789~012}
\email{schmidiger.kevin@hotmail.com}
%\extrainfo{informacion adicional}
\photo[64pt][0.4pt]{picture}
%\quote{Die aktuelle Version meines Lebenslauf auf \href{https://github.com/kschmidi/cv}{\textbf{Github}}}
%\quote{The only thing that interferes with my learning is my education. - Albert Einstein}
\social[github]{kschmidi}

\begin{document}

\makecvtitle

% ------------------------------------------------------------------------------------------------------------
% Section 1
% ------------------------------------------------------------------------------------------------------------
\section{Ausbildung}
\cventry{2013--heute}
{Bachelor of Science FHO in Computer Science}
{\hfill\bfseries Rapperswil}
{\newline{}\href{http://www.hsr.ch/}{HSR Hochschule f"ur Technik Rapperswil}}
{\hfill\textit{Note: -}}
{Vollzeit vermischt mit Teilzeit am Schluss}

\cventry{2012--2013}
{Berufsmaturit"at (Technische Richtung)}
{\hfill\bfseries Winterthur}
{\newline{}\href{http://www.bbw.ch/}{BBW Berufsbildungsschule Winterthur}}
{\hfill\textit{Note: 4.8}}
{Vollzeit}

\cventry{2007--2011}
{Berufslehre Elektroinstallateur EFZ}
{\hfill\bfseries Z"urich}
{\newline{}\href{htttp://www.tbz.ch/}{TBZ Technische Berufsschule Z"urich}}
{\hfill\textit{Note: 5.0}}
{Lehre}

% ------------------------------------------------------------------------------------------------------------
% Second 2
% ------------------------------------------------------------------------------------------------------------
\section{Erfahrungen}
\cventry{2016-217}
{Praktikum - \href{https://www.avaloq.ch/}{Avaloq AG}}{}
{\newline{}6 Monatiges Praktikum. Mitarbeit im  agilen Team der Tools Innovation (Eclipse Plug-ins). Kleinere Bugfixes und Features, sowie testing.}{}
{\href{https://www.avaloq.ch/}{Avaloq AG} \hfill Zu"rich}

\cventry{2016}
{\href{http://sinv-56012.edu.hsr.ch/ccglator}{Bachelorarbeit -} \href{https://www.youtube.com/watch?v=CeNdN5KlPGc}{Plug-in f"ur} \href{https://github.com/isocpp/CppCoreGuidelines/blob/master/CppCoreGuidelines.md}{C++ Core Guideline unterst"uztung}}{}
{\newline{}Einhaltung der Rules: C.ctor: Constructors, assignments, and destructors mithilfe von Codeanalyse und Refactoring}{\hfill \textbf{Note: 5.5}}
{IFS Institut f"ur Software \hfill\href{http://www.hsr.ch/}{HSR Hochschule f"ur Technik Rapperswil}}

\cventry{2015}
{\href{https://eprints.hsr.ch/479/1/aliextor.pdf}{\textbf{Studienarbeit -}} \href{https://www.youtube.com/watch?v=QNrb_LbuYmI}{Plug-in} f"ur C++(1y) Refactoring (Type Alias Extraction)}{}
{\newline{}Einarbeitung in Plug-in developement, CDT, IFS Testingframework u. "A}{\hfill \textbf{Note: 5.5}}
{IFS Institut f"ur Software \hfill\href{http://www.hsr.ch/}{HSR Hochschule f"ur Technik Rapperswil}}

\cventry{2015}
{Coach an der \href{http://appquest.hsr.ch/}{AppQuest} 2015}{}
{\newline{}Betreuung von Teams w"ahrend 12 Wochen}{}
{AppQuest \hfill\href{http://www.hsr.ch/}{HSR Hochschule f"ur Technik Rapperswil}}

% ------------------------------------------------------------------------------------------------------------
% Section 3
% ------------------------------------------------------------------------------------------------------------
\section{Technologien}
%\subsection{Sehr gute Kenntnisse}
\cventry{\bfseries +++}{Java, C/C++11, C\#, MacOSX, Linux, Android, Jenkins, Git, \LaTeX}
{}{}{}{}
%\subsection{Gute Kenntnisse}
\cventry{\bfseries ++}{}{}{Bash, Assembler, JavaScript, SQL, iOS, HTML, CSS, Play, JSF, Maven, ANTLR, PostgreSQL}{}{}
%\subsection{Grundkenntnisse}
\cventry{\bfseries +}{}{}{Compilerbau, Swift, PHP, Haskell, Bootstrap, Spring, Windows Azure, Docker, AngularJS, Oracle, MySQL}{}{}

% ------------------------------------------------------------------------------------------------------------
% Section 4
% ------------------------------------------------------------------------------------------------------------
\section{Sprachen}
%\subsection{Sehr gute Kenntnisse}
\cventry{\bfseries +++}{Deutsch, \textnormal{Englisch}}
{}{}{}{}
%\subsection{Grundkenntnisse}
\cventry{\bfseries +}{Franz"osisch}
{}{}{}{}

% ------------------------------------------------------------------------------------------------------------
% Section 5
% ------------------------------------------------------------------------------------------------------------
\section{Freizeit}
\subsection{Interessen}
\cventry{}{Eishockey \textnormal{(NLA/NHL)}, Astronomie, \textnormal{Allg. Verst"andnis der Physik und Mathematik}, Programmiersprachen \textnormal{(Aufbau bzw. Compilerbau)}, Elektrotechnik \textnormal{(Programmierung von Hardware)}, \textnormal{Allg.} Technik}
{}{}{}{}
\subsection{Hobbies}
\cventry{}{Lesen, Gitarre spielen, Sport}
{}{}{}{}

%\nocite{*}
%\bibliographystyle{plain}
%\bibliography{publications} 

\end{document}